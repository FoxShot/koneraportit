\documentclass[main.tex]{subfiles}

\begin{document}
\thispagestyle{empty}
\begin{tabular}[t]{l}
Kone: HP TC4400\\
Käyttis: Lubuntu 15.04
\end{tabular}
\hfill 26.10.2015

{\section{Näytönkierto}}

Ohjelmalla \texttt{xrandr} voidaa kääntää näyttöä antamalla tarkennin \texttt{-o} esim.

\begin{lstlisting}
xrandr -o right
\end{lstlisting}

kääntää näytön oikealle. Vastaavasti \texttt{left} vasemmalle, \texttt{inverted} ylösalaisin ja \texttt{normal} palauttaa.

\begin{lstlisting}
xinput list
\end{lstlisting}

näyttää listauksen syötelaitteista. Listassa laitte "\texttt{Serial Wacom Tablet stylus}" on näyttökynä, jonka toiminta tulee myös kääntää

\begin{lstlisting}
xsetwacom set "Serial Wacom Tablet stylus" rotate ccw
\end{lstlisting}

esim. kääntää orientaation oikealle (ccw 'counter clock wise'), \texttt{cw} kääntää vasemmalle, \texttt{half} ylösalaisin ja \texttt{none} palauttaa.

\texttt{/sys/devices/platform/hp-wmi/tablet} antaa 0 kun läppäri-mode ja 1 kun tablet-mode. Näistä koottiin skripti, joka valvoo koska moodi vaihtuu ja pyöräyttää sen mukaan näytön ympäri.

%\verbatimtabinput{./scripts/20151026.sh}
\end{document}