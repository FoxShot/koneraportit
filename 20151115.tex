\documentclass[main.tex]{subfiles}

\begin{document}
\thispagestyle{empty}
\begin{tabular}[t]{l}
Kone: extendedshadow\\
Käyttis: ubuntu 14.04 LTS
\end{tabular}
\hfill 15.11.2015

{\scshape\Large{Texmaker 4.1}}
Asennettiin texlive

\begin{lstlisting}
sudo apt-get install texlive
\end{lstlisting}

Ei saatu haluttua tulosta, joten asennettiin koko paketti

\begin{lstlisting}
sudo apt-get install texlive-full
\end{lstlisting}

Huomattiin, että tarkoitus oli tehdä asennus texmakerille joten

\begin{lstlisting}
sudo apt-get install texmaker
\end{lstlisting}

Texmakerissa on nyt asetuksissa nappi, jolla ulos tulevat tiedostot siirretään alikansioon "build". Kuitenkin myös pdf-tiedostot menevät tähän kansioon, joten tämä vastaa hieman \texttt{-output-directory=} määritettä pdflatexiin.
\end{document}