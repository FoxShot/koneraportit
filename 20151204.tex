\documentclass[main.tex]{subfiles}

\begin{document}
\thispagestyle{empty}
\begin{tabular}[t]{l}
Kone: HP TC4400\\
Käyttis: Lubuntu 15.04
\end{tabular}
\hfill 4.12.2015

{\section{Netflix ja Silverlight}}

Asennettiin pipelight reposta

\begin{lstlisting}
sudo add-apt-repository ppa:pipelight/stable
sudo apt-get update
sudo apt-get -y install pipelight-multi samba-dev
\end{lstlisting}

Enabloitiin silverlight-liitännäinen

\begin{lstlisting}
sudo pipelight-plugin --enable silverlight
\end{lstlisting}

Käynnistettiin Firefox ja \texttt{wine} asensi silverlighting. Asennettiin \texttt{UAControl}-liitännäinen Firefoxiin \texttt{addons.mozilla.org} ja huijattiin \texttt{www.netflix.com} luulemaan konetta Windows-koneeksi lisäämällä \texttt{UAControl}:iin sivun asetus

\begin{lstlisting}
Mozilla/5.0 (Windows NT 6.1; rv:29.0) Gecko/20131011 Firefox/29.0
\end{lstlisting}

Tämän jälkeen luotiin vielä Firefox-liitäineinen \texttt{pipelight}:lle

\begin{lstlisting}
sudo pipelight-plugin --create-mozilla-plugins
\end{lstlisting}
\end{document}