\documentclass[main.tex]{subfiles}

\begin{document}
\thispagestyle{empty}
\begin{tabular}[t]{l}
Kone: HP TC4400\\
Käyttis: Lubuntu 15.10
\end{tabular}
\hfill 6.12.2015

{\section{Lubuntu ongelmat}}

\texttt{libdrm2}:n poistaminen aiheutti useiden tärkeiden pakettien poiston. Kone ei enään käynnistynyt oikein. Yritettiin korjata päivittämällä 15.10. Bootatessa virhe \texttt{missing configuration file. keyword: path}. Tabulaattorilla selvisi mahdolliset komennot ja kirjoittamalla 

\begin{lstlisting}
live
\end{lstlisting}

saatiin live-session käynnistettyä, mutta päivitys ei toiminut ja ainut mikä toimi oli puhdas asennus. Ensin tuli ottaa dekryptatut käyttäjätiedostot talteen. Live-sessiossa mountattiin kovalevy ja ajettiin

\begin{lstlisting}
sudo ecryptfs-recover-private
\end{lstlisting}

Avattiin \texttt{pcmanfm} järjestelmän valvojana, jotta tiedostoihin päästiin käsiksi.

\begin{lstlisting}
gksu pcmanfm
\end{lstlisting}

Lubuntu asennuksen jälkeen ei boottia. Vain vilkkuva kursori mustalla ruudulla. Piti asentaa \texttt{grub} erilleen tälle asemalle.

\begin{lstlisting}
sudo mmount /dev/sdb1 /mnt
sudo grub-install --root-directory=/mnt /dev/sdb
\end{lstlisting}

Kolmannen osapuolen ajureita ei asennettu, joten firefox flash plugin asennettiin

\begin{lstlisting}
sudo apt-get install flashplugin-installer 
\end{lstlisting}

Texmaker ja suomenkielen paketti

\begin{lstlisting}
sudo apt-get install texmaker texlive-lang-european
\end{lstlisting}

Siirrettiin tallennetut tiedostot ja koitettiin asentaa \texttt{steam}, mutta virheilmoitus \texttt{failed to load steamui.so}, joten nimettiin uudelleen kansio ja ajettiin asennus uudelleen

\begin{lstlisting}
mv .local/share/Steam{,.old} 
\end{lstlisting}

Kopioitiin vanhat pelitiedostot uuteen asennukseen

\begin{lstlisting}
cp .local/share/Steam{.old,}/common 
\end{lstlisting}

Peli ei näkynyt käynnistäessä asennettuna, mutta asennuksen jälkeen säilytti kuitenkin tallennukset. Tämä saattoi kuitenkin johtua siitä, että tallennukset olivat pilvessä. Myöskin \texttt{.config}-polun kopiointi uuteen asennukseen säilytti \texttt{textmaker}:n asetukset.
\end{document}