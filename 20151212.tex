\documentclass[main.tex]{subfiles}

\begin{document}
\thispagestyle{empty}
\begin{tabular}[t]{l}
Kone: extendedshadow\\
Käyttis: Ubuntu 14.04 LTS
\end{tabular}
\hfill 12.12.2015

{\scshape\Large{Videoeditointi, työpöytäjumi ja gif-konversio}}

Asennettiin \texttt{openshot}

\begin{lstlisting}
sudo apt-get install openshot
\end{lstlisting}

Käytettiin ohjelmaa videon leikkaamiseen ja pätkästä luotiin png-kuvasarja. Työpöytä päätti kuitenkin jumittaa ja se piti käynnistää uudelleen kirjautumalla ulos painaen \texttt{ctrl+alt+F1}. Kirjauduttiin komentorivillä sisään ja käynnistettiin työpyötä uudelleen

\begin{lstlisting}
sudo service lightdm restart
\end{lstlisting}

Kuvakansiossa ajettiin \texttt{convert}

\begin{lstlisting}
convert -delay 5 -loop 0 *.png animation.gif
\end{lstlisting}

Kuvien tulee olla järjestyksessä, siis loppunumeron muotoa 01 tai 001 riippuen kuvien määrästä.

{\Large{Edit:}}

Video saadaan helpommin convertoitua gif-muotoon luomalla ensin sarja jpg-tiedostoja seuraavasti

\begin{lstlisting}
mkdir frames
ffmpeg -i <video> -r 5 'frames/frame-%03d.jpg'
\end{lstlisting}

missä 5 on fps eli kääntäen 20 ms delay convertointia varten. \texttt{ffmpeg} asenetaan repositorystä

\begin{lstlisting}
sudo add-apt-repository ppa:mc3man/trusty-media
sudo apt-get update
sudo apt-get install ffmpeg
\end{lstlisting}

gif muodostuu komennolla

\begin{lstlisting}
convert -delay 20 -loop 0 ./frames/*.jpg animation.gif
\end{lstlisting}

\end{document}