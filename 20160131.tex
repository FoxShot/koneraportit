\documentclass[main.tex]{subfiles}

\begin{document}
\thispagestyle{empty}
\begin{tabular}[t]{l}
Kone: HP TC4400\\
Käyttis: Lubuntu 15.04
\end{tabular}
\hfill 31.1.2016

{\scshape\Large{Oikeinkirjoituksen tarkistus}}

Käytettään \texttt{aspell} ohjelmaa tarkistamaan Latex-dokumentteja. Valitsimella \texttt{-t} ilmoitetaan, että kyse on tex-dokumentista. Suomen sanakirja valitaan \texttt{--master=fi}. Koska kyse on vanhasta tiedostosta tulee vielä ilmoittaa kirjaisinkoodaus \texttt{--encoding=ISO-8859-1}. Ilmoitetaan vielä haluttu dokumentti valitsimella \texttt{-c} ja kokonaisuudessaan komento on

\begin{lstlisting}
  aspell -t --encoding=ISO-8859-1 --master=fi -c <tex-tiedosto>
  asf
\end{lstlisting}

\end{document}