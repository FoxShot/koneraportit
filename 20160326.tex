\documentclass[main.tex]{subfiles}

\begin{document}
\thispagestyle{empty}
\begin{tabular}[t]{l}
Kone: Lenovo X61 tablet\\
Käyttis: Lubuntu 15.10
\end{tabular}
\hfill 26.3.2016

{\section{\texttt{youtube-dl} ja vakioasetukset}}

Luotiin tiedosto tunnusten ja salasanojen tallentamista varten ja luotiin sille käyttöoikeudet

\begin{lstlisting}
gedit ~/.netrc
chmod a-rwx,u+rw ~/.netrc
\end{lstlisting}

Tunnukset ja salasanat kirjoitettiin tiedostoon muodossa

\begin{lstlisting}
machine <extractor> login <login> password <password>
\end{lstlisting}

missä \texttt{extractor} on esimerkiksi \texttt{youtube}. Listauksen mahdollisista extractoreista saa komennolla

\begin{lstlisting}
youtube-dl --list-extractors
\end{lstlisting}

Lisäksi luotiin konfiguraatiotiedosto \texttt{youtube-dl}:ää varten.

\begin{lstlisting}
gedit ~/configuration/youtube-dl/config
\end{lstlisting}

Tiedostoon voidaa laittaa automaattisesti käytettävät valitsimet. Jotta ohjelma lataisi tunnukset ja salasanat tiedostosta, lisättiin konfiguraatiotiedostoon rivi

\begin{lstlisting}
--netrc
\end{lstlisting}

Tärkeintä on kuitenkin valita kansio minne tiedostot ladataan

\begin{lstlisting}
--output ~/Videos/%(title)s.%(ext)s
\end{lstlisting}
\end{document}