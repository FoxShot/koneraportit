\documentclass[a4paper, 12pt, twoside]{artikel3}
\usepackage[utf8]{inputenc}
\usepackage[finnish]{babel}
\usepackage{geometry}
\usepackage[dvipsnames, table]{xcolor}
\usepackage{float}
\usepackage{mdframed}

\geometry{
a4paper,
left=35mm,
right=25mm,
top=10mm,
bottom=20mm
}

\floatstyle{plain}
\newfloat{program}{H}{lop}
\floatname{program}{Program}

%\def\[{\begin{program}\begin{mdframed}[backgroundcolor=BurntOrange!30]\begin{Verbatim}}
%\def\]{\end{Verbatim}\end{mdframed}\end{program}}

\renewcommand{\[}{\begin{program}\begin{mdframed}[backgroundcolor=BurntOrange!30]\verb}
\renewcommand{\]}{\end{mdframed}\end{program}}

\usepackage{moreverb}
\usepackage{fancyvrb}

\begin{document}
\thispagestyle{empty}
\begin{tabular}[t]{l}
Kone: Lenovo X61 tablet\\
Käyttis: Lubuntu 15.10
\end{tabular}
\hfill 27.3.2016

{\scshape\Large{Python ja gdata-paketti}

Ongelmia oli löytää moduuli \texttt{tlslite.utils} kun asennettiin GitHubista \texttt{gdata-python-client-master}. Lataamalla paketti \texttt{gdata-2.0.18} ja asentamalla saatiin oikeat tiedostot. Kuitenkin nämä tuli siirtää polusta oikeaan

\[
  sudo cp -r /usr/local/lib/python2.7/dist-packages/ /usr/lib/python2.7/dist-packages/
\]
\end{document}