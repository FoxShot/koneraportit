\documentclass[main.tex]{subfiles}


\begin{document}
\thispagestyle{empty}
\begin{tabular}[t]{l}
Kone: Lenovo X61\\
Käyttis: Lubuntu 16.04
\end{tabular}
\hfill 10.2.2017

\section{Python \texttt{youtube-dl} moduuli}

Oli ongelmia saada moduuli toimimaan suoraan python-ohjelmasta. Kokeiltiin 

\begin{lstlisting}
  sudo pip install youtube-dl
\end{lstlisting}

mutta se asensi python version 2.5. Saatiin toimimaan lataamalla github versio ja ajamalla

\begin{lstlisting}
  sudo python3 setup.py install
\end{lstlisting}

kansiossa, mikä asensi python 3.5 version. Kun tämä saatiin toimimaan oli ongelma löytää key 'url' joka olisi pitänyt saada extract-info komennolla. Kuitenkin komento tuotti ensin listan laaduista ['formats'] mistä piti valita yksi ja sitten ['url']. Lopulta päätettiin etsiä yksi quality-id ja tulostaa sen url.

\begin{lstlisting}
  for laatu in formats:
  	print(laatu['format'])
  	if laatu['format_id'] == '43':
  		print(laatu['url'])
  		video = laatu['url']\end{lstlisting}

\end{document}