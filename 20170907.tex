\documentclass[main.tex]{subfiles}


\begin{document}
\thispagestyle{empty}
\begin{tabular}[t]{l}
Kone: extendedshadow\\
Käyttis: Ubuntu 16.04
\end{tabular}
\hfill 7.9.2017

\section{MySQL setuppi}

Asennettiin MySQL workbench .deb-paketista ja \texttt{mysql} komentoriviltä

\begin{lstlisting}
  sudo apt-get install mysql
  sudo mysql_secure_installation
\end{lstlisting}

Toiminta voidaan tarkistaa komennolla

\begin{lstlisting}
  sudo service mysql status
\end{lstlisting}

Ohjelmaa ei kuitenkaan voitu käyttää root-käyttäjänä. Jotta saataisiin käyttökelpoinen database tuli luoda ensin sellainen ja sille käyttäjä oikeuksinee

\begin{lstlisting}
  sudo mysql
  > CREATE DATABASE komponentit;
  > USE komponentit;
  > CREATE USER 'user'@'host' INDENTIFIED BY 'salasana';
  > GRANT ALL PRIVILEGES ON komponentit.* to user@host;
  > FLUSH PRIVILEGES;
\end{lstlisting}

Vakiona oleva salasanavaatimuksen taso on MEDIUM (1), mikä vaatii ison- ja pienenkirjaimen; numeron ja erikoismerkin. Salasanan on oltava vähintää 8 merkkiä. Salasanan voi vaihtaa komennolla

\begin{lstlisting}
  SET PASSWORD FOR 'user'@'host' = password('salasana')
\end{lstlisting}
\end{document}