\documentclass[main.tex]{subfiles}


\begin{document}
\thispagestyle{empty}
\begin{tabular}[t]{l}
Kone: extendedshadow\\
Käyttis: Ubuntu 16.04
\end{tabular}
\hfill 10.1.2018

\section{32-bittinen \texttt{wine}, Steam ja Hearts of Iron III}
\texttt{wine} ei päivittynyt Ubuntu repositystä oikeaan versioon, joten ensin poistettiin väärä versio ja asennettiin paketti virallisesta reposta.

\begin{lstlisting}
  sudo apt-get remove wine winetricks
  sudo dpkg --add-architecture i386
  wget -nc https://dl.winehq.org/wine-builds/Release.key
  sudo apt-key add Release.key
  sudo apt-add-repository https://dl.winehq.org/wine-builds/ubuntu/
  sudo apt-get update
  sudo apt-get install --install-recommends winehq-stable
  sudo apt-get install winetricks
\end{lstlisting}

Hearts of Ironia varten tulee mahdollisesti myös asentaa \texttt{.NET 2.0} ja \texttt{DirectX}, yms.

\begin{lstlisting}
  winetricks vcrun2005 directx9 dotnet20 quartz
\end{lstlisting}

32-bittistä emulaatiota varten tehtiin kansio ja ohjattiin se käyttämään 32-arkkitehtuuria (kansiota ei saa olla ennalta olemassa vaan \texttt{winecfg} luo sen).

\begin{lstlisting}
  WINEARCH=win32 WINEPREFIX=~/.wine32 winecfg
\end{lstlisting}

Vakio prefiksi voidaan asettaa komennolla

\begin{lstlisting}
  export WINEPREFIX=~/.wine32
\end{lstlisting}

muuten tulee käynnistää ohjelmat käyttäen komentoa

\begin{lstlisting}
  WINEPREFIX=~/.wine32 wine <ohjelma.exe>
\end{lstlisting}

Steamia käynnistäessä teksti ei näkynyt ollenkaan, mutta lisäämällä optio komentoon saatiin se näkymään

\begin{lstlisting}
  WINEPREFIX=~/.wine32 wine Steam.exe --no-dwrite
\end{lstlisting}

Hearts of Iron käynnistyi kummallisella resoluutiolla joten muokattiin sen \texttt{Settings.txt}-tiedostosta (löytyy pelin juurihakemistosta) kohta

\begin{lstlisting}
  size=
  {
    x=1680
    y=1050
  }
\end{lstlisting}
\end{document}